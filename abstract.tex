\documentclass[./master]{subfiles}
\begin{document}
%\begin{abstract}
%\chapter*{概要}
\mbox{}

%\vspace{0.5\baselineskip}
\begin{center}\LARGE\textbf{概要}\end{center}
\vspace{\baselineskip}

  宇宙元素合成において,${}^{12}\mathrm{C}$原子核は3つの$\alpha$粒子から直接合成される(トリプルアルファ反応).
  トリプルアルファ反応では主に3$\alpha$崩壊閾値近傍に位置する3$\alpha$共鳴状態である$0_2^+$ (\SI{7.65}{\mega\electronvolt})
  状態 (Hoyle状態) を経由する.大半の$0_2^+$状態は3つの$\alpha$粒子に崩壊するが,稀に$\gamma$線を放出して脱励起することで,
  安定した${}^{12}\mathrm{C}$原子核となる.
  トリプルアルファ反応は${}^{12}\mathrm{C}$より重い元素を合成するための戸口反応であり,
  宇宙元素合成において最も重要な原子核反応の一つである.
  しかし,$\rho = \SI{e6}{\gram\per\centi\metre}$のような高密度環境下では,
  $\gamma$崩壊に加え,他の粒子との非弾性散乱による脱励起が増加しトリプルアルファ反応を劇的に促進することが指摘されている.
  特に中性子は電荷を持たずクーロン力の効果を受けないため,
  脱励起を促進する効果が大きいと考えられている.

  脱励起の反応率の計算には,${}^{12}\mathrm{C}$が中性子との散乱により
  ${}^{12}\mathrm{C} (0_2^+)$へ励起する断面積が必要となる.
  特に,$0_2^+$状態へ励起させることができる中性子エネルギーの
  閾値付近 ($E_{\text{lab}} = \SI{8.3}{\mega\electronvolt}$) における断面積が重要となる.
  しかし,$E_{\text{lab}} = \SI{8.3}{\mega\electronvolt}$付近におけるg.s. $\rightarrow$ $0_2^+$状態の
  断面積は測定されていない.
  そこで,我々は中性子ビームを用いて${}^{12}\mathrm{C} (0_2^+)$へ励起し,
  ${}^{12}\mathrm{C} (0_2^+)$状態から崩壊した3つの低エネルギー$\alpha$粒子を測定することで,
  崩壊元の状態を特定しg.s. $\rightarrow$ $0_2^+$状態反応の断面積を決定することを計画している.
  このためには,3つの$\alpha$粒子を全て測定するための大きな立体角を多い,
  低エネルギー$\alpha$粒子を効率的に検出することのできる検出器が必要となる.
  この要求を満たす検出器にMAIKo TPC がある.

  MAIKo TPC では,検出器のガス中を通過した荷電粒子のトラックが画像として検出される.
  検出器中でのトラックの長さと方向から荷電粒子のエネルギーと運動量を決定するため,
  画像から$\alpha$粒子のトラックを正しく抽出することが必要となる.
  MAIKo TPC で検出されるトラックの分解能が検出ガスの種類によって大きく変わる.
  そこで,どのような検出ガスが測定に適しているか,
  その時の検出効率は十分か,$0_2^+$状態を識別するのに十分な分解能が達成できるかを
  現実的な実験条件を仮定して検討する必要がある.

  本研究では,MAIKo TPC で用いる検出ガスの候補を複数選出し,
  $\alpha$線源を用いた性能試験を行った.
  また,それらの検出ガスについて,
  中性子との散乱で${}^{12}\mathrm{C}$原子核が3つの$\alpha$粒子に崩壊するイベントの画像を,
  シミュレーションによって生成した.
  さらに,シミュレーションで生成した画像に対して解析を行い,
  検出効率,エネルギー分解能,角度分解能を評価した.
  シミュレーションによる検討の結果,
  \SI{100}{\hecto\pascal}の\isoButaneHydro を検出ガスに用いれば,
  計画中の実験を遂行するのに十分な検出効率および${}^{12}\mathrm{C}$の励起エネルギー分解能を実現できることが推定された.
%\end{abstract}
\end{document}
