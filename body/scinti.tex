\documentclass[../master]{subfiles}

%\graphicspath{{../eps/}}

\begin{document}

\chapter{中性子検出器}
\section{液体シンチレータ}
中性子は電荷をもっていないため、中性子と検出器中の粒子とが反応して
生成した荷電粒子を検出することで、中性子を検出する。
中性の検出にはNE213/BC501 液体シンチレータを用いる。
液体シンチレータでは主に水素原子の原子核である陽子が散乱されることで検出される。

\section{読み出し回路}
CAEN V1742 を用いて中性子検出器から得られる信号を取得した。
CAEN V1742 は入力信号の波形をそのまま取得することができるモジュールである。
信号の取得周波数は5 GHz から 750 MHz である。

\section{n-$\gamma$弁別}
液体シンチレータを用いた測定では中性子だけでなく背景$\gamma$線も検出してしまう。
そのため、中性子と$\gamma$線の識別が必須となる。
中性子と$\gamma$線では液体シンチレータの発光の波形が異なることが知られている。
図\ref{fig::pulse_shape_n_gamma}に中性子と$\gamma$線の波形の違いの模式図を示す。
\begin{figure}
  \centering
%  \includegraphics[clip, width=0.8\columnwidth]{}
  \caption{液体シンチレータから得られる中性子および$\gamma$線の波形。}
  \label{fig::pulse_shape_n_gamma} 
\end{figure}
中性子の方がテールが長く引いた波形となる。
2つの積分区間を用いて波形を積分することで、このような2つの波形を識別する。
図\ref{fig::integ_region}に2つの積分区間の模式図を示す。
1つは波形のピークを含むような区間、1つは主にテールの部分を積分するような区間である。


\section{キャリブレーション}
\section{SCINFUL-CG による中性子の検出効率}
検出器中に入射した中性子が他の粒子と反応しない場合は検出されない。
そのため、検出器に入射した中性子の絶対数を求めるためには検出効率を知る必要がある。
液体シンチレータの検出効率を求める計算コードにSCINFUL-CG~\cite{scinful-cg}がある。

\end{document}
