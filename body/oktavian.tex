\documentclass[master]{subfiles}

\begin{document}

\chapter{OKTAVIAN}
\section{OKTAVIAN}
大阪大学強力14MeV中性子工学実験装置 (OKTAVIAN) によって生成した単色中性子ビームを用いて
${}^{12}\rm{C}(n,n'){}^{12}\rm{C}(0_{2}^{+})$の断面積の測定を行う。
中性子の発生には
\begin{equation}
  t(d,{}^{4}\rm{He})n
\end{equation}
反応を用いる。
この反応を用いることにより、
およそ$14\rm{MeV}$の単色中性子を発生させることが可能となる。

コッククロフト・ワルトン型加速装置を用いることで、
デューテリウムを加速しトリチウム標的に照射する。
OKTAVIANには連続照射ラインとパルスラインの2つのビームラインがある。
パルスラインでは$1\rm{kHz}$--$2\rm{MHz}$のパルス状の
ビームを照射することができる。
ビーム電流は時間平均で$6.67\rm{p\mu A}$である。
連続照射ラインでは連続的にビームを小差hすることができ、
ビーム電流は$6.67\rm{pmA}$である。
ビームの時間情報を用いて解析を行うことが可能となるが、
パルスビームラインのトリチウム標的は実験室内にあり、
コリメートされていない中性子を用いることになる。
この場合、実験室の壁などに反跳した中性子がバックグランドになるため、
本実験には適していない。
そのため、この実験では連続照射ラインを使用する。
連続照射ラインを用いる場合、
重照射室においてデューテリウムビームをトリチウム標的に照射し、
大実験室との間にある穴から中性子ビームの取り出しを行う。

\subsection{MAIKo架台}
中性子のバックグラウンドを低減させるため、
実験装置は可能な限り取り出し穴に近づける必要がある。
しかし、取り出し穴のあるは階段上にあるため、
階段の上に実験装置を設置することとなる。

\section{中性子ビーム}
\subsection{コリメータ}
\subsection{ビーム量およびエネルギー}

\end{document}
