\documentclass[../master]{subfiles}

\begin{document}

\chapter*{謝辞}
本研究は多くの方のご助力により成立しています.
指導教官である川畑貴裕教授には,研究の進め方,文章の書き方,発表方法などの多くのことをご指導いただきました.
また,実験の合間にキャッチボールやソフトボールに連れ出していただくことで,
心身ともに健康に研究を行うことができました.
大阪と京都の二重生活でお忙しいにも関わらず,多くの時間を私への指導に当てていただき大変感謝しております.

OKTAVIAN の村田勲教授には,中性子発生装置のことや実験に向けたアドバイスなど多くのご助力を頂き,大変感謝しております.

RCNP の古野達也さんと村田求基さんには,RCNP でMAIKo TPC のテストをしている際に,
TPC の先輩として測定方法やシミュレーションの方法など多くのアドバイスとご助力をいただきました.
%また,食事などに誘って頂き,一人で作業をしていることが多かったため,とても感謝しております.
岡本慎太郎くんには,MAIKo TPC のテストや息抜きの卓球を一緒に行い,多くの時間をともに過ごしました.
一人では大変な作業を手伝って頂き,大変助かりました.
稲葉健斗さんには,京都にいるMAIKo TPC のエキスパートとして多くの相談に乗っていただきました.
特にガスやMAIKo TPC の取扱について,なれない私に丁寧にご教授いただきました.
また,一緒にフィットネスに行くことで,健康的な生活を送る一助となりました.
ありがとうございました.
藤川祐輝さん,大阪大学の坂梨公亮くんには,OKTAVIAN での測量など人手が必要な作業をお手伝いいただき,大変助かりました.
土方佑斗くん,延與紫世さんには,解析の手伝いをして頂き大変助かりました.

同期の関屋涼平くん,原田健志くん,藤井涼平くん,古田悠稀くんとは食事の際にお互いの研究について,
気軽に意見を言い合うことができ大変楽しい時間を過ごしました.
研究室の永江知文教授,成木恵准教授,銭廣十三准教授,村上哲也講師,後神利志助教授,先輩方は
常に研究の進捗を気にかけてくださいました.

最後に,今まで私を支えて頂いた家族や友人に対して深く感謝の意を申し上げます.

\end{document}
