\documentclass[../master]{subfiles}
\begin{document}

\chapter{中性子ビームを用いた実験の結果}
\section{本実験の目的}
2020/02/25--2020/02/28にOKTAVIAN にて中性子ビームを用いた実験を行った.
この実験では,中性子ビームを入射した際に得られる飛跡画像とシミュレーションによる飛跡画像の比較と
MAIKo TPC を用いた測定により決定した断面積と既知の断面積の比較を目的としている.

\section{実験セットアップ}
ビームライン,MAIKo TCP および中性子検出器の配置は図\ref{fig::alignment}の通りである.
\\写真と模式図

\subsection{ビームライン}
写真

\subsection{MAIKo TPC}
写真

\subsection{中性子検出器}
写真

\section{測定結果}
\subsection{得られた飛跡画像}

\subsection{${}^{12}\mathrm{C}(\mathrm{n},\mathrm{n}'){}^{12}\mathrm{C}(0_2^+)$反応の断面積}

\end{document}
