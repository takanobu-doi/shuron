\documentclass[../master]{subfiles}

\begin{document}

\chapter{まとめと今後の展望}
%\section{まとめ}
本研究では${}^{12}\mathrm{C}(\mathrm{n}, \mathrm{n}'){}^{12}\mathrm{C}^{\text{Hoyle}}$反応の断面積測定のための
MAIKo TPC の検出ガスの選定を行った.
測定では${}^{12}\mathrm{C}^{\text{Hoyle}}$から放出される3つの$\alpha$粒子を検出しなければならない.
$\alpha$粒子のエネルギーの決定はMAIKo TPC で取得されたトラックの長さから行うため,
$\alpha$粒子がMAIKo TPC の有感領域で停止するとが必要となる.
また,トラックが短くなるとトラックを識別できなくなるため,
物質厚が大きすぎないことが必要となる.
トラックが太いと複数のトラックを識別できなくなるため,
電子の拡散効果が小さいガスが必要となる.
このような要求を満たす検出ガスを決定した.

$\alpha$粒子が適当な飛距離で効率的に停止する観点から,
\Methane~(\SI{50}{\hecto\pascal}),\MethaneHydro~(\SI{100}{\hecto\pascal}),\MethaneHerium~(\SI{100}{\hecto\pascal}),
\isoButaneHydro~(\SI{100}{\hecto\pascal}),\isoButaneHerium~(\SI{100}{\hecto\pascal}) の
5種類が検出ガスの候補として選出された.
MAIKo TPC がタイムウィンドウは\SI{10}{\micro\second},ドリフト方向の長さが\SI{140}{\milli\metre}であるため,
ドリフトスピードが\SI{0.014}{\milli\metre\per\nano\second}となる必要がある.
このドリフトスピードを実現するドリフト電場で測定を行う.

$\alpha$線源を用いてトラックを測定し,そのトラックを再現するようなシミュレーションを作成した.
シミュレーションで${}^{12}\mathrm{C}(\mathrm{n}, \mathrm{n}'){}^{12}\mathrm{C}^{\text{Hoyle}}$反応のトラックを生成し,
eye-scan による解析を行った.
解析の結果より,検出効率,エネルギー分解能などの観点で\MethaneHydro と\isoButaneHydro が適していることが分かった.
\MethaneHydro と\isoButaneHydro を比較すると${}^{12}\mathrm{C}$の含まれる量が\isoButaneHydro の方が$4/3$多い.
そのため,検出ガスは\isoButaneHydro に決定した.

%\section{本測定に向けて}
2/25--2/28の4日間でOKTAVIAN で\SI{14}{\mega\electronvolt} の中性子を用いた測定を行う予定である.
本研究で決定した検出ガスを用いて測定を行う.

\end{document}
