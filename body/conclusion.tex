\documentclass[../master]{subfiles}

\begin{document}

\chapter{まとめと今後の展望}
%\section{まとめ}
本研究では${}^{12}\mathrm{C}(\mathrm{n}, \mathrm{n}'){}^{12}\mathrm{C} (0_2^+)$反応の断面積測定のための
実験条件を検討した.
${}^{12}\mathrm{C}(\mathrm{n}, \mathrm{n}'){}^{12}\mathrm{C} (0_2^+)$反応において,
崩壊してできる$\alpha$粒子の持つエネルギーが数百\si{\kilo\electronvolt}と小さいことが分かった.
また,広い角度に放出されることも分かった.
そこで,${}^{12}\mathrm{C} (0_2^+)$から放出される3つの低エネルギー$\alpha$粒子をすべて検出するために,
低エネルギーの荷電粒子を大立体角で検出できるMAIKo TPC を用いて測定行うことを決定した.
約\SIrange{8.35}{15}{\mega\electronvolt}の範囲で断面積の中性子エネルギー依存性を測定するが,
まずは検証実験として単色エネルギーで生成可能な\SI{14}{\mega\electronvolt}の中性子を用いた測定を行う.
そのため,\SI{14}{\mega\electronvolt}の中性子と${}^{12}\mathrm{C}$との反応に主眼をおいて検討を進めた.

MAIKo TPC では$\alpha$粒子のエネルギーを取得されたトラックの長さから決定するため,
$\alpha$粒子がMAIKo TPC の有感領域で停止することが必要となる.
しかし,トラックが短くなるとトラックを正しく識別できなくなるため,
適切な物質厚であることが必要となる.
要求を満たすガスとしてガスとして,\Methane~(\SI{50}{\hecto\pascal}),\MethaneHydro~(\SI{100}{\hecto\pascal}),
\MethaneHerium~(\SI{100}{\hecto\pascal}),\isoButaneHydro~(\SI{100}{\hecto\pascal}),
\isoButaneHerium~(\SI{100}{\hecto\pascal}) の5種類を候補とした.
検出ガスの種類によっては電子の拡散効果が大きく,
荷電粒子のトラックが太く検出される.
太いトラックでは3つの$\alpha$粒子のトラックを正しく識別できないため,
拡散の効果が小さいことが求められる.
拡散の効果において,\MethaneHydro と\isoButaneHydro が有力であることが分かった.
また,実際の測定で取得されるであろうトラックをシミュレーションにより生成し,
実際に解析を行うことで検出ガスの評価を行った.
評価の結果,\MethaneHydro と\isoButaneHydro では大きな優劣の差は見られなかった.
そこで,体積当たりの${}^{12}\mathrm{C}$の含有量の多い\isoButaneHydro を検出ガスとして決定した.

\isoButaneHydro を検出ガスに用いることで,
${}^{12}\mathrm{C}$の$0_2^+$状態を識別するのに十分な分解能を達成できることが分かった.
また,検出器中で3つの$\alpha$粒子が停止する割合は\SI{48.2}{\percent},
それらのイベントから正しく$\alpha$粒子のトラックを抽出できる割合は\SI{87}{\percent}であることが分かった.
これらを考慮して,${}^{12}\mathrm{C}(\mathrm{n}, \mathrm{n}'){}^{12}\mathrm{C} (0_2^+)$反応の収量を見積もると,
24時間で59.6~events であると期待される.

%\section{本測定に向けて}
2020年2/25--28の4日間でOKTAVIAN で測定を行う予定である.
本研究で決定した検出ガスを用いて測定を行い,
シミュレーション計算との詳細な比較を行う.

\end{document}
