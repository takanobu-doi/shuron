\documentclass[../master]{document}

\begin{document}

\chapter{解析}
%\section{液体シンチレータの解析}
%\subsection{FADCの生データ}
%光電子増倍管から取得した波形データの一例を示す。
%
%\subsection{波形弁別 (n-$\gamma$ discrimination)}
%液体シンチレータのデータの中には中性子による信号とガンマ線による信号とが含まれている。
%この2つの波形には違いがあるので、その波形の違いから識別することができる。
%
%\subsection{中性子のレート}
%中性子線源から放出される中性子の量は時間とともに変化する。
%
\section{解析の概要}
%\subsection{機械学習}
%これまではHough 変換を使って解析を行ってきたが、
%高速に処理をするためにニューラルネットワークを用いた解析方法を開発した。

\section{eye-scan}
\subsection{eye-scanによる選別}
\subsection{eye-scanによる励起エネルギーの再構成}

\section{結果}
\subsection{検出効率}
\subsection{検出効率の角度依存性}
\subsection{エネルギー分解能}
\subsection{角度分解能}
\subsection{励起エネルギー分解能}

\section{検出ガスの決定}

\end{document}
