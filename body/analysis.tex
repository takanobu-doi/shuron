\documentclass[../master]{subfiles}

\begin{document}

\chapter{解析}
%\section{液体シンチレータの解析}
%\subsection{FADCの生データ}
%光電子増倍管から取得した波形データの一例を示す。
%
%\subsection{波形弁別 (n-$\gamma$ discrimination)}
%液体シンチレータのデータの中には中性子による信号とガンマ線による信号とが含まれている。
%この2つの波形には違いがあるので、その波形の違いから識別することができる。
%
%\subsection{中性子のレート}
%中性子線源から放出される中性子の量は時間とともに変化する。
%
\section{解析の概要}
MAIKo TCP の解析では背景事象の除去とトラック情報の抽出の2つが必要となる。
検出ガスには${}^{12}{\rm C}$だけでなく、陽子や${}^{4}{\rm He}$が含まれる。
そのため、中性子と陽子、${}^{4}{\rm He}$との散乱事象を取り除く必要がある。
中性子と${}^{12}{\rm C}$との散乱に対してトラックの情報を抽出する。
トラックの情報は中性子と${}^{12}{\rm C}$とが散乱した座標、
$\alpha$粒子が停止した点である。
anode image から$y, z$座標を、cathode image から$x, y$座標を決定することができる。
$x, z$座標は$\mu$-PIC のstrip 数に400 $\mu$mをかけることで決定できる。
$y$座標はTPC では荷電粒子が通過した位置から読み出し面に到達するまでの時間として測定される。
そのため、anode image, cathode image のclock にドリフトスピードをかけることで$y$座標を決定できる。
このようにして決定したanode image, cathode image の座標を合わせることで、
3次元の座標を決定することができる。

散乱点と停止点の座標から粒子が飛行した方向ベクトルと距離が決定される。
同じ粒子であれば、飛行距離から運動エネルギーが一意に決まる。
図\ref{fig::range_to_ene_alpha}に$\rm CH_{4}$ 50 hPa 中での荷電粒子の飛行距離と運動エネルギーの対応を示す。
\begin{figure}
  \centering
  \includegraphics[clip, width=0.8\columnwidth]{range_to_ene.eps}
  \caption[$\rm CH_{4}$ 50 hPa 中での荷電粒子の飛行距離と運動エネルギー。]
          {$\rm CH_{4}$ 50 hPa 中での荷電粒子の飛行距離と運動エネルギー。
          }
  \label{fig::range_to_ene_alpha}
\end{figure}
この対応関係から粒子の運動エネルギーを決定する。
粒子の運動エネルギーを$T$、単位方向ベクトルを$(dx, dy, dz)$とすると、
4元運動量は
\begin{equation}
  p =
  \begin{pmatrix}
    E \\ p_{x} \\ p_{y} \\ p_{z}
  \end{pmatrix}
  =
  \begin{pmatrix}
    T + m \\ \sqrt{(T+m)^2 + m^2} dx \\ \sqrt{(T+m)^2 + m^2} dy \\ \sqrt{(T+m)^2 + m^2} dz
  \end{pmatrix}
  \label{eq::momentum_vector}
\end{equation}
となる。
決定した3つの$\alpha$粒子の4元運動量から${}^{12}{\rm C}$の4元運動量を再構成できる。
このようにして求めた4元運動量から${}^{12}{\rm C}$のエネルギー、散乱角度、励起エネルギーを求めることができる。

%\subsection{機械学習}
%これまではHough 変換を使って解析を行ってきたが、
%高速に処理をするためにニューラルネットワークを用いた解析方法を開発した。

\section{eye-scan}
本研究ではTPC の解析を人間の目 (eye-scan) で行った。
eye-scanでは、トラックの本数の識別と散乱点、停止点の抽出を行う。
ここではトラックの本数が3本であるイベントを${}^{12}{\rm C}({\rm n},{\rm n}')3\alpha$イベントとした。
本研究では${}^{12}{\rm C}({\rm n},{\rm n}')3\alpha$イベントに対して解析を行った。

\subsection{検出効率}
${}^{12}{\rm C}({\rm n},{\rm n}')3\alpha$イベントであっても、
各$\alpha$粒子のエネルギーや放出角度、トラックの太さによっては3つのトラックを区別することができない場合がある。
そこで、正しくトラックが3本と認識できる割合を評価する。
\ref{seq::triple_alpha_simulation}

\subsection{検出効率の角度依存性}
\subsection{エネルギー分解能}
\subsection{角度分解能}
\subsection{励起エネルギー分解能}

\section{検出ガスの決定}

\end{document}
