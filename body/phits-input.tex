\documentclass[../master]{subfiles}
\begin{document}
\chapter{PHITS のインプットファイル}
\label{chap::phits-input}
PHITS のインプットファイルを以下に示す.
ポリエチレンのコリメータの場合のシミュレーションである.
\begin{quote}
  \setlength{\baselineskip}{12pt}
\begin{verbatim}
[ T i t l e ]
simulation for neutron collimator

[ P a r a m e t e r s ]
 icntl    =  0
 itall    =  1
 maxcas   =  5000000
 maxbch   =  50
 file(6)  =  phits.out

[ S o u r c e ]
  s-type  =   1
    proj  =   neutron
     dir  =   all
      r0  =   0.
      z0  =  -146.4
      z1  =  -146.4
      e0  =   14.

[ M a t e r i a l ]
 mat[1] $ Air
         N 8 O 2
 mat[2] $ Polyethylene
         C 2 H 4
 mat[3] $ Concrete
         O  -0.52  Si -0.325 Ca -0.06
         Na -0.015 Fe -0.04  Al -0.04
 mat[4] $ Acrylic
         C 5 O 2 H 8
 mat[5] $ Methane
         C 1 H 4

[ S u r f a c e ]
$ colimator
 100   cz   5.5
 101   cz   1.
 102   pz   0.
 103   pz   100.
$ wall
 104   rpp  -100. 100. -100. 100. 0. 100.
$ frange
 110   cz   5.5
 111   pz   102.
 112   pz   104.
$ detector
 120   pz   110.
$ room
 200   rpp  -100. 100. -100. 100. -200. 300.

[ C e l l ]
$ collimator
 100   2   -0.9  -100 +101 +102 -103
$ wall
 200   3   -2.5     -104 +100
$ frange
 300   4   -1.18    -110 +111 -112
$ detector
 400   5   -0.000717 -110 +112 -120
$ room
 1000  1   -0.0012  -200 #100 #200 #300 #400
$ void
 2000  -1           +200

[ T - C r o s s ]
    title =   Energy distribution in r-z mesh (front)
     mesh =   r-z
   r-type =   1
       nr =   3
              0. 1. 5.5 10
   z-type =   1
       nz =   0
              102.
   e-type =   2
       ne =   150
     emin =   0.
     emax =   15.
     unit =   2 
     axis =   eng
     file =   cross_eng_f.out
   output =   f-curr
     part =   all neutron
    gshow =   1
   epsout =   1

[ T - C r o s s ]
    title =   Energy distribution in r-z mesh (rear)
     mesh =   r-z
   r-type =   1
       nr =   2
              0. 1. 5.5
   z-type =   1
       nz =   0
              104.
   e-type =   1
       ne =   3
              0. 13.5 14.5 20.
     unit =   2 
     axis =   eng
     file =   cross_eng_r.out
   output =   f-curr
     part =   all neutron
    gshow =   1
   epsout =   1

[ T - C r o s s ]
    title =   Posion distribution in xyz mesh (front)
     mesh =   xyz
   x-type =   2
       nx =   100
     xmin =  -10.
     xmax =   10.
   y-type =   2
       ny =   100
     ymin =  -10.
     ymax =   10.
   z-type =   1
       nz =   0
              102.
   e-type =   2
       ne =   1
     emin =   0.
     emax =   20.
     unit =   1 
     axis =   xy
     file =   cross_xy_f.out
   output =   f-curr
     part =   all neutron
    gshow =   1
   epsout =   1

[ T - C r o s s ]
    title =   Posion distribution in xyz mesh (rear)
     mesh =   xyz
   x-type =   2
       nx =   100
     xmin =  -10.
     xmax =   10.
   y-type =   2
       ny =   100
     ymin =  -10.
     ymax =   10.
   z-type =   1
       nz =   0
              104.
   e-type =   2
       ne =   1
     emin =   0.
     emax =   20.
     unit =   1 
     axis =   xy
     file =   cross_xy_r.out
   output =   f-curr
     part =   all neutron
    gshow =   1
   epsout =   1

[ T - 3 D s h o w ]
   output =   3
 material = 4
            2 3 4 5
       x0 =   0.
       y0 =   0.
       z0 =   0.
    e-the =   40.
    e-phi =   45.
    e-dst =   500.
    l-the =   150.
    l-phi =   30.
    l-dst =   80.
    w-wdt =   80.
    w-hgt =   100.
    w-dst =   300.
   heaven =   x
     line =   2
   shadow =   2
    resol =   2
     file =   3dshow.out
    title =   Check geometry
   epsout =   1

[Mat Name Color]
  mat    name          color
  1      Air           pastelblue
  2      Polyethylene  red#yellow
  3      Concrete      camel
  4      Acrylic       blue

[ E n d ]
\end{verbatim}
\end{quote}
\end{document}
