\chapter{Experimental setup}
本実験では中性子ビームを${}^{12}\rm{C}$標的に入射し、
崩壊した3$\alpha$を測定する。
大阪大学強力14MeV中性子工学実験装置 (OKTAVIAN) を
用いて中性子ビームを生成し、
$\mu$-PIC based Active target for Inverse Kinematics .
(MAIKo) TPC を用いて荷電粒子を測定する。
\section{OKTAVIAN}
OKTAVIANは大阪大学大学院工学研究科にある中性子発生装置である。
中性子の発生には
\begin{equation}
  t(d,{}^{4}\rm{He})n
\end{equation}
反応を用いる。
この反応を用いることにより、
およそ$14\rm{MeV}$の単色中性子を発生させることが可能となる。

コッククロフト・ワルトン型加速装置を用いることで、
デューテリウムを加速しトリチウム標的に照射する。
OKTAVIANには連続照射ラインとパルスラインの2つのビームラインがある。
パルスラインでは$1\rm{kHz}$--$2\rm{MHz}$のパルス状の
ビームを照射することができる。
ビーム電流は時間平均で$6.67\rm{p\mu A}$である。
連続照射ラインでは連続的にビームを小差hすることができ、
ビーム電流は$6.67\rm{pmA}$である。
ビームの時間情報を用いて解析を行うことが可能となるが、
パルスビームラインのトリチウム標的は実験室内にあり、
コリメートされていない中性子を用いることになる。
この場合、実験室の壁などに反跳した中性子がバックグランドになるため、
本実験には適していない。
そのため、この実験では連続照射ラインを使用する。
連続照射ラインを用いる場合、
重照射室においてデューテリウムビームをトリチウム標的に照射し、
大実験室との間にある穴から中性子ビームの取り出しを行う。


\section{MAIKo TPC}
\subsection{MAIKo TPC とは}
TPC では荷電粒子の飛跡を3次元的に検出することが可能である。

ここら辺はomegのproceedingsに書いた文章をマネする。

${}^{12}\rm{C}$が崩壊して生成される$\alpha$粒子は
主に数百$\rm{keV}$のエネルギーを持って放出される。
そのため、このような低エネルギー粒子を検出することができる
セットアップが必要となる。

\subsection{検出ガスの決定}
標的には炭素の含まれる炭化水素を用いる。
この実験では低エネルギーの荷電粒子の飛跡を検出するため、
飛跡が比較的長くなるエネルギーロスが小さいガスが適する。
そこで、質量数が最も小さいメタン ($CH_{4}$) を用いた。
また、ガスの圧力によって飛跡の長さが変化する。
そこで、$\alpha$粒子の検出効率がよくなるガス圧を求めた。

ガス圧はシミュレーションによって決定した。
${}^{12}\rm{C}$と中性子との散乱をKondoらの実験で求められた
微分断面積の角度分布を用いて再現し、
散乱後に${}^{12}\rm{C}$が$Ex = 7.65\rm{MeV}$に励起し、
${}^{12}\rm{C}\rightarrow{}^{8}\rm{Be}+{}^{4}\rm{He}\rightarrow{}^{4}\rm{He}\times 3$
と崩壊する過程を考えた。
この時、$\alpha$粒子が持つエネルギーの分布は図\ref{fig::alphaenergydist}のようになる。
このような粒子に対して
\begin{enumerate}
\item
  MAIKoの有感領域内 ($102.4\rm{mm}\times 102.4\rm{mm}\times 140\rm{mm}$) で停止する
\item
  飛跡の長さが$20\rm{mm}$以上である
\end{enumerate}
という条件の時に検出可能とすると、検出効率の圧力依存は図\ref{fig::alphaefficiency}のようになる。
図\ref{fig::alphaefficiency}より、$75\rm{hPa}$付近が最も検出効率が高いことが分かる。
そこで、$50\rm{hPa}$、$75\rm{hPa}$、$100\rm{hPa}$の3通りでのオペレートを決定した。

\subsection{ドリフトスピード}
TPCの特性上、ドリフト電場方向のアクセプタンスは電子のドリフトspeedに依存する。
ドリフトケージの大きさ ($140\rm{mm}$) を可能な限り使用するためには、
MAIKo TPC の時間アクセプタンス ($10.24\rm{\mu s}$) で$140\rm{mm}$となるようなドリフトスピード
 ($140\rm{mm}/10.24\rm{\mu s} \sim 0.0135\rm{mm/ns}$) に調整する必要がある。

%ドリフト速度の決定方法は30 degree 方向に$\alpha$線源から$\alpha$を出して、
%その飛跡がデータ上でどう見えるかで決定する。
%ドリフト速度の時間依存性も見た。

\subsection{MAIKo架台}
中性子のバックグラウンドを低減させるため、
実験装置は可能な限り取り出し穴に近づける必要がある。
しかし、取り出し穴のあるは階段上にあるため、
階段の上に実験装置を設置することとなる。


\section{中性子カウンター (液体シンチレータ)}
\subsection{キャリブレーション}
\subsection{波形弁別}
\subsection{検出効率}

\section{中性子カウンター (金属箔)}

\section{中性子ビーム}
\subsection{コリメータ}
\subsection{ビーム量およびエネルギー}
