\chapter{Introduction}
\section{宇宙での元素合成過程}
宇宙での元素合成過程において、${}^{12}\rm{C}$は重要な役割を持っている。
${}^{4}\rm{He}$までの原子核は水素の燃焼過程により合成される。
しかし、質量数が5と8の安定な原子核が存在しないため、
それらが合成されても直ちにより軽い核へ崩壊してします。
そのため、質量数が8よりも思い原子核を合成することが出来ない。
この問題はFred Hoyle が予言した$3\alpha$の共鳴状態 (Hoyle状態) に
よって解決された。
${}^{12}\rm{C}$原子核は3つの$\alpha$の共鳴状態として、
トリプル$\alpha$反応によって生成される。
生成された${}^{12}\rm{C}^{*}$のほとんどは再び$3\alpha$へと崩壊するが、
一部は${}^{12}\rm{C}$の基底状態へ$\gamma$崩壊する。
${}^{12}\rm{C}$原子核はこのように生成され、
さらに重い原子核が生成されていく。

\section{トリプルアルファ反応}
近年、高温高密度領域では陽子や中性子との散乱による脱励起で
崩壊幅が増加することが示唆されている。~\cite{hotdensemedium}
これにより$2_{2}^{+}$や$0_{1}^{+}$への脱励起が増加し、
トリプルアルファ反応が加速される。

${}^{12}\rm{C}$と中性子の反応レートは
\begin{equation}
  r = N_{n}N_{{}^{12}\rm{C}}\braket{\sigma v}\rm{cm}^{-3}\rm{sec}^{-1}
  \label{eq::r}
\end{equation}
で与えられる。
ここで、$N_{n}$は中性子の個数密度、
$N_{{}^{12}\rm{C}}$は${}^{12}\rm{C}$の個数密度を表す。
$\sigma$は始状態 (g.s.または$Ex = 4.44\rm{MeV}$) から
Hoyle状態 ($Ex=7.65\rm{MeV}$) へ励起する全断面積であり、
$v$は2粒子の相対速度である。
相対速度がMaxwell分布に従うとすると、
${}^{12}\rm{C}$の中性子非弾性散乱では
\begin{equation}
  \braket{\sigma v}_{nn'} =
  \left(\frac{8}{\pi\mu}\right)^{1/2}\left(\frac{1}{kT}\right)^{^3/2}
  \int^{\inf}_{0}E'\sigma_{n,n'}(E')\exp(-E'/kT)dE'
  \label{eq::sigmann'}
\end{equation}
となる。
我々が考える反応は上記の逆過程なので、
\begin{equation}
  \braket{\sigma v}_{n'n} = \left(\frac{2I+1}{2I'+1}\right)
  \exp(-Q/kT)\braket{\sigma v}_{nn'}
  \label{eq::sigman'n}
\end{equation}
となる。
ここで、$I$およびIh$I'$は始状態 (g.s.または$Ex = 4.44\rm{MeV}$)
および終状態 (Hoyle状態) のスピンである。
$Q$は$-7.654\rm{MeV}$ (g.s.からの場合) または
$-3.215\rm{MeV}$ ($Ex = 4.44\rm{MeV}$からの場合) となる。
${}^{12}\rm{C}$の中性子非弾性散乱による脱励起の寿命は
\begin{equation}
  \tau_{n'n}({}^{12}\rm{C}^{\rm{Hoyle}}) =
  (N_{n}\braket{\sigma v}_{n'n})^{-1} \rm{sec}
  \label{eq::tau}
\end{equation}
となる。

$\gamma$崩壊の寿命 ($\tau_{\gamma} = 1.710\times10^{-13}$) との比を
$R$とすると、
\begin{equation}
  R = 6.557\times10{-6}\times\rho_{n}T_{9}^{-1.5}\rm{C}_{\rm{spin}}
  \int~{\inf}_{0}\sigma_{nn'}(E)(E-Q)\exp(-11.605E/T_{9})dE
  \label{eq::R}
\end{equation}
と表される。
$E$はc.m.系のエネルギー、$\rho_{n}$は中性子の密度 ($\rm{g/cm3}$)、
$\sigma_{nn'}(E')$は断面積 ($\rm{mb}$)である。
$\rm{C}_{\rm{spin}}$はg.s.からの場合1、
$Ex = 4.44\rm{MeV}$からの場合5となる。
式(\ref{eq::R})からわかるように、中性子によって脱励起する過程は
特に温度に大きく依存する。
Rと温度の依存性を図\ref{fig::R}に示す。

ただし、この計算に必要な中性子と${}^{12}\rm{C}$との断面積のデータは
図\ref{fig::crosssection_pres}にあるように、
Hoyle状態へのものが不足している。

\section{${}^{12}\rm{C}(n,n'){}^{12}\rm{C}^{*}$の断面積}


\section{本研究の目的}
高温高密度環境では粒子と散乱することによって脱励起する
過程の寄与が大きくなる。
しかし、中性子との散乱断面積のデータは不足している。
そこで、${}^{12}\rm{C}^{*}$から崩壊する低エネルギーの3$\alpha$を
直接測定することで、${}^{12}\rm{C}(n,n'){}^{12}\rm{C}^{*}$の断面積を
決定する。
最終的には幅広い中性子の入射エネルギーでの測定が必要となるが、
まずは$14\rm{MeV}$の中性子での測定を行う。

本論文では低エネルギー$\alpha$粒子の測定方法や
得られたデータの形跡手法、断面積の決定方法について述べる。
