\chapter{検出器}
\section{MAIKo TPC}
\subsection{MAIKo TPC とは}
TPC では荷電粒子の飛跡を3次元的に検出することが可能である。

ここら辺はomegのproceedingsに書いた文章をマネする。

${}^{12}\rm{C}$が崩壊して生成される$\alpha$粒子は
主に数百$\rm{keV}$のエネルギーを持って放出される。
そのため、このような低エネルギー粒子を検出することができる
セットアップが必要となる。

\section{検出ガスの決定}
標的には炭素の含まれる炭化水素を用いる。
この実験では低エネルギーの荷電粒子の飛跡を検出するため、
飛跡が比較的長くなるエネルギーロスが小さいガスが適する。
そこで、質量数が最も小さいメタン ($\rm{CH}_{4}$) を用いた。
また、ガスの圧力によって飛跡の長さが変化する。
そこで、$\alpha$粒子の検出効率がよくなるガス圧を求めた。

ガス圧はシミュレーションによって決定した。
${}^{12}\rm{C}$と中性子との散乱をKondoらの実験で求められた
微分断面積の角度分布を用いて再現し、
散乱後に${}^{12}\rm{C}$が$Ex = 7.65\rm{MeV}$に励起し、
${}^{12}\rm{C}\rightarrow{}^{8}\rm{Be}+{}^{4}\rm{He}\rightarrow{}^{4}\rm{He}\times 3$
と崩壊する過程を考えた。
この時、$\alpha$粒子が持つエネルギーの分布は図\ref{fig::alphaenergydist}のようになる。
このような粒子に対して
\begin{enumerate}
\item
  MAIKoの有感領域内 ($102.4\rm{mm}\times 102.4\rm{mm}\times 140\rm{mm}$) で停止する
\item
  飛跡の長さが$20\rm{mm}$以上である
\end{enumerate}
という条件の時に検出可能とすると、検出効率の圧力依存は図\ref{fig::alphaefficiency}のようになる。
図\ref{fig::alphaefficiency}より、$75\rm{hPa}$付近が最も検出効率が高いことが分かる。
そこで、$50\rm{hPa}$、$75\rm{hPa}$、$100\rm{hPa}$の3通りでのオペレートを決定した。

\section{ドリフトスピード}
TPCの特性上、ドリフト電場方向のアクセプタンスは電子のドリフトspeedに依存する。
ドリフトケージの大きさ ($140\rm{mm}$) を可能な限り使用するためには、
MAIKo TPC の時間アクセプタンス ($10.24\rm{\mu s}$) で$140\rm{mm}$となるようなドリフトスピード
($140\rm{mm}/10.24\rm{\mu s} \sim 0.0135\rm{mm/ns}$) に調整する必要がある。

\section{拡散効果}
\section{ガスの種類及び圧力の決定}

%ドリフト速度の決定方法は30 degree 方向に$\alpha$線源から$\alpha$を出して、
%その飛跡がデータ上でどう見えるかで決定する。
%ドリフト速度の時間依存性も見た。

%\section{中性子カウンター (液体シンチレータ)}
%\subsection{キャリブレーション}
%\subsection{波形弁別}
%\subsection{検出効率}
%
%\section{中性子カウンター (金属箔)}
%
